% !Mode::"XeLaTeX:UTF-8"
\documentclass[a4paper]{article}
\usepackage{ctex}
\usepackage{engord}
\usepackage{pifont}
\usepackage{siunitx}
\usepackage{chngcntr}
\usepackage{enumitem,mdwlist}
\usepackage{fontspec,listings}
\usepackage[perpage]{footmisc}
\usepackage{multicol,pdflscape}
\usepackage[margin=1in]{geometry}
\usepackage[numbered,framed]{matlab-prettifier}
\usepackage{amsmath,amssymb,amsthm,empheq,mathtools,amsthm,mathrsfs}
\usepackage{graphicx,epstopdf,subcaption,float,wrapfig}
\usepackage{array,booktabs,threeparttable,hhline,multirow}
\usepackage{cleveref}
\usepackage{tikz}
\usetikzlibrary{decorations.markings}
\sisetup{list-separator={, },list-final-separator={, },list-pair-separator={, },separate-uncertainty,range-phrase={\ensuremath{\sim}}}
\newcommand\crefpairgroupconjunction{和}\newcommand\crefmiddlegroupconjunction{、}\newcommand\creflastgroupconjunction{和}
\newcommand\crefrangeconjunction{至}\newcommand\crefrangepreconjunction{}\newcommand\crefrangepostconjunction{}
\newcommand\crefpairconjunction{和}\newcommand\crefmiddleconjunction{、}\newcommand\creflastconjunction{和}
\renewcommand\thefootnote{\ding{\numexpr171+\value{footnote}}}
\crefname{listing}{程序}{程序}\Crefname{listing}{程序}{程序}
\crefname{equation}{式}{式}\Crefname{equation}{式}{式}
\crefname{section}{节}{节}\Crefname{section}{节}{节}
\crefname{figure}{图}{图}\Crefname{figure}{图}{图}
\crefname{table}{表}{表}\Crefname{table}{表}{表}
\renewcommand{\lstlistingname}{程序}
\AtBeginDocument{\counterwithin{lstlisting}{section}}
\counterwithin{figure}{section}
\counterwithin{table}{section}
\setmonofont{Consolas}
\catcode`\。= 13
\def。{.}
\title{《数字信号处理》大作业}
\author{涂金正, 电41, 2014010914}
\date{2016年12月5日}

\newcommand{\jj}{\mathrm{j}}
\newcommand{\Sa}{\mathrm{Sa}}
\newcommand{\dd}{\mathrm{d}}
\newcommand{\question}[1]{\renewcommand{\thesection}{习题~#1}\section}
\newcommand{\questionx}[1]{\question{#1}{(题目略)}}
\newenvironment{solution}[1][\unskip]{\renewcommand{\qedsymbol}{}\proof[#1 解]}{\endproof}
\newcommand{\gen}{./Generated}
\newcommand{\matlab}[1]{\lstinputlisting[style=Matlab-editor,basicstyle=\ttfamily\small,numbers=none,caption={\texttt{#1}}]{#1}}
\newcommand{\largefigure}[1]{\begin{center}\includegraphics[width=\linewidth]{\gen/#1}\end{center}}
\newcommand{\smallfigure}[1]{\begin{center}\includegraphics[width=0.7\linewidth]{\gen/#1}\end{center}}
\newcommand{\result}[1]{\lstinputlisting[basicstyle=\ttfamily\tiny,breaklines]{\gen/#1}}
\DeclareMathOperator{\acosh}{arccosh}
\DeclareMathOperator{\asinh}{arcsinh}
\newcommand{\thefigures}[2]{%
\begin{figure}[H]
  \centering
  \begin{subfigure}[b]{0.3\linewidth}
    \centering
    \includegraphics[width=\linewidth,trim=0 0 0 0,clip,keepaspectratio]{\gen/#2-phasor.eps}
    \caption{正序相量}
    \label{fig:#2-phasor}
  \end{subfigure}
  \begin{subfigure}[b]{0.3\linewidth}
    \centering
    \includegraphics[width=\linewidth,trim=0 0 0 0,clip,keepaspectratio]{\gen/#2-frequency.eps}
    \caption{频率}
    \label{fig:#2-frequency}
  \end{subfigure}
  \begin{subfigure}[b]{0.3\linewidth}
    \centering
    \includegraphics[width=\linewidth,trim=0 0 0 0,clip,keepaspectratio]{\gen/#2-phasors}
    \caption{三相相量}
    \label{fig:#2-phasors}
  \end{subfigure}
  \caption{#1}
\end{figure}}

\makeatletter
\def\@maketitle{%
  \newpage
  \null
  \begin{center}%
  \let \footnote \thanks
    {\LARGE \@title \@date \par}%
    {\large
      \begin{tabular}[t]{c}%
        \@author
      \end{tabular}\par}%
  \end{center}%
  \par
  \vskip 1.5em
  }
\fi
\makeatother

\begin{document}
\maketitle
\section{任务一:PMU测量的实现与性能测试}
\subsection{PMU测量的Matlab实现}
编写了Matlab函数\lstinline"pmu",
传入不同的\lstinline"h0"和\lstinline"h1"参数,就可以实现基于不同的滤波器的PMU测量。
详见附录。

\subsection{矩形窗对标准波形的性能测试曲线}
\thefigures{矩形窗,$\Delta f=\SI{0}{\hertz}$}{rect-naive-s0}
\thefigures{矩形窗,$\Delta f=\SI{+1}{\hertz}$}{rect-naive-sp1}
\thefigures{矩形窗,$\Delta f=\SI{-1}{\hertz}$}{rect-naive-sm1}
\thefigures{矩形窗,$\Delta f=\SI{+2}{\hertz}$}{rect-naive-sp2}
\thefigures{矩形窗,$\Delta f=\SI{-2}{\hertz}$}{rect-naive-sm2}
\thefigures{矩形窗,AM-FM}{rect-naive-amfm}

\subsection{三角窗对标准波形的性能测试曲线}
\thefigures{三角窗,$\Delta f=\SI{0}{\hertz}$}{trig-naive-s0}
\thefigures{三角窗,$\Delta f=\SI{+1}{\hertz}$}{trig-naive-sp1}
\thefigures{三角窗,$\Delta f=\SI{-1}{\hertz}$}{trig-naive-sm1}
\thefigures{三角窗,$\Delta f=\SI{+2}{\hertz}$}{trig-naive-sp2}
\thefigures{三角窗,$\Delta f=\SI{-2}{\hertz}$}{trig-naive-sm2}
\thefigures{三角窗,AM-FM}{trig-naive-amfm}

\subsection{TVE和FE最大值}
见附录。

\section{任务二:两种滤波器方案的相量测量性能指标}
根据附录所示计算结果,三角窗在所有测试波形下各种误差都要小于矩形窗。
这是由于三角窗的频响特性(见\cref{fig:freqz})下降更为剧烈,在低频段波动更小导致的。
\begin{figure}[htbp]
  \centering
  \includegraphics[width=0.7\linewidth,trim=0 0 0 0,clip,keepaspectratio]{\gen/h0.eps}
  \caption{\label{fig:freqz} 三角窗和矩形窗的频响特性}
\end{figure}

综上所述,三角窗的谐波抑制、抗噪声性能比矩形窗好,
矩形窗的动态响应性能比三角窗好。

\section{任务三:实际录波数据检验}
\subsection{矩形窗对实际录波数据的性能测试曲线}
\thefigures{矩形窗,实际数据电压}{rect-naive-volt}
\thefigures{矩形窗,实际数据电流}{rect-naive-curr}
\subsection{三角窗对实际录波数据的性能测试曲线}
\thefigures{三角窗,实际数据电压}{trig-naive-volt}
\thefigures{三角窗,实际数据电流}{trig-naive-curr}

结合附录计算结果,可见这两种方法相比于PMU设备,还是非常na\"{i}ve,有着非常大的误差,
尤其是对于频率的测量;
这是由于\lstinline"h1"滤波器的阶数太低,把高频噪声分量无限制放大造成的。
另外,实际PMU设备同时考虑到了电压和电流的频率,比起前述na\"{i}ve的PMU算法,可以互相校正,更利于稳定输出。

\section{任务四:M类应用中的滤波器设计与分析}

\section{附录}
Matlab程序:
\matlab{gen.m}
\matlab{pmu.m}
\matlab{err.m}
\matlab{scheduler.m}
运行结果:
\result{data.txt}

\end{document}
